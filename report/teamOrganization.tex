     % How was the team organized
     %Diyana text
     % Interaction as developers
     %Sara text
     % Organization of the repositories
     %Who text?
     % Branching strategies
     %Diyana text
     % Development process and tools
     %Diyana text
     Within the team, we decided to create a GitHub organization which allows us to keep a collection of GitHub repositories in a shared resource. Within the group we all had different roles which allowed us to work more efficiently, this means that whilst some group members might not seem active within GitHub, that does not mean that they were not active within the project since things like the azure setup does not show up within GitHub
     
     \subsubsection{Interaction as developers}
     We had weekly meetings in the University where we would quickly update each other with our progress and then divide the new tasks. In some cases, we needed to spend more time to finish/plan a task, so we would schedule an extra meeting in the University, or we will make a Teams call if not all members are available to meet in person. Sometimes, we would do “pair-programming”, so we would pick a task (not necessarily a programming one) and we would research the topic together and compare our findings. 
     
    \subsubsection{Branching strategies}
    We decided to follow the Centralized flow to manage our Github repositories. That means we have one Main branch and we create a subbranch from it per feature. When the feature is done, we merge the subbranch with the master branch. That is because one person is typically responsible for a specific project, meaning only one person will be working on the project at a time and thus we do not need a complex GitHub flow with many different branches. Nevertheless, we did use a Develop branch for the web application but that is mostly for learning purposes – push to Develop, make sure there are no conflicts, and the app works as expected and then finally merge that with the Main branch.
    
    \subsubsection{Development process and tools}
    To manage our tasks, we used Trello and Github. In Trello we were assigning mostly non-programming tasks like Code Quality assessment, Risk Analysis, Software licensing etc. Intuitively, we were tracking the state of each task by moving from a "To Do column", to a "Done Done" column that contained “reviewed” tasks. For the programming tasks, we used Github issues to close on a Pull request that contains the required updates. 
     
     